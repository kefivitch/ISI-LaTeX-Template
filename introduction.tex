\chapter*{Introduction générale}
\addcontentsline{toc}{chapter}{Introduction générale} % to include the introduction to the table of content
\markboth{Introduction générale}{} %To redefine the section page head
Durant ces dernières années, un changement fondamental est survenu dans la stratégie des entreprises dans lesquelles plusieurs activités se basent sur l'utilisation des applications informatiques. En effet, un logiciel qui est limité à la réalisation des besoins fonctionnels de ses utilisateurs n'est plus satisfaisant. Autres exigences sont alors introduites dont la performance, la maintenabilité et l'ergonomie des applications. Dans ce contexte que plusieurs entreprises ont opté pour la migration de leurs applications vers des technologies récentes.\\ \newline
Avec Essentiels2020\cite{essentiels2020}, Orange affirme une ambition centrale, exigeante et forte : faire vivre à chacun de son client une expérience incomparable. Et devenir l’opérateur numéro 1 en recommendation auprés de trois quarts des 263 millions de clients dans les 28 pays Orange. La qualité de service qu’ils perçoivent doit être parfaite à chaque instant de la journée, chez eux ou sur leur lieu de travail, en mobilité sur leurs trajets du quotidien ou dans les transports.\\ \newline
En raison de la satisfaction de ses clients, Orange groupe cherche à avoir des systèmes d'information non seulement qui répondent à ses besoins variables mais encore qui s'adaptent rapidement aux évolutions requises par son environnement actif. Ainsi Sofrecom doit présenter des solutions pour répondre aux besoins de son partenaire et plus important client.\\ \newline
En particulier aux clients professionnels et aux petites et moyennes entreprises, Orange  France propose deux solutions différentes pour garantir la satisfaction clientèle, pour répondre à leurs questions et leurs réclamations. Ces deux solutions représentent deux métiers différents : un centre d’appel en composant 3901 ou une boutique Orange.\\ \newline
La différence entre ces deux métiers implique l’interaction avec le client d’où le client doit être présent chez la boutique Orange pour passer une demande ou une réclamation ou il peut interagir avec le centre d’appel  via un appel téléphonique au numéro 3901. Par contre, il ont le même client cible (les petites et moyennes entreprises) et une importante partie commune qui sert à gérer les clients, gérer les applications, etc...
\\\newline
C’est dans ce cadre que se situe l’élaboration de notre sujet de projet de fin d’études au sein de Sofrecom Tunisie qui consiste à la refonte de le portail Panoramix pour intégrer la nouvelles population des boutiques Orange avec la population existante qui représente le centre d’appel 3901 afin d’augmenter la fiabilité, centraliser les données et l'historique des réclamations client, l’efficacité de l’effort humain et faciliter les tâches pénibles au sein de l’établissement Orange France.\\ \newline
Le présent rapport est structuré en six chapitres.
\begin{itemize}
	\item Le premier chapitre concerne la présentation de l'organisme d’accueil, contexte du sujet, la problématique ainsi que le diagnostic de l’existant et la méthodologie utilisée.
	\item Le deuxième chapitre présente les concepts théoriques utiles pour l’élaboration de notre projet.
	\item Le troisième chapitre est consacré à le premier sprint intitulé“ Migration de Panoramix“. Nous commençons d’abord par une étude comparative des technologies. Après, nous présentons l'enchaînement d’automatisation des tests de non régression.
	\item Le quatrième chapitre décrit le deuxième sprint réservé au “gestion des utilisateurs et leurs rôles”.
	\item Le  cinquième chapitre  dédié au sprint 2 qui décrit l’intégration de module Boutique dans notre application.
	\item Le dernier chapitre ,dédié au dernier sprint, est consacré à la gestion des catégories des PEF et la gestion des PEF
\end{itemize}
Finalement, nous clôturons ce travail réalisé pendant ce projet par une conclusion générale, tout en décrivant quelques perspectives.
