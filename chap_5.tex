\chapter{Sprint 2 – Module Boutique}
	
\section*{Introduction}
Dans ce chapitre, nous allons nous intéresser à le module Boutique. Par la suite, nous nous intéresserons à l’aspect conceptuel et fonctionnel de ce sprint.

\section[Backlog sprint]{Backlog sprint}
Dans le backlog du sprint, nous présenterons deux parties :  le but du sprint et les user stories.

\subsection[But de sprint]{But de sprint}
En suivant le même principe que le sprint précédent, nous commencerons par définir l'objectif du sprint. Par conséquent, l'objectif de ce sprint est l'intégration de module Boutique et l’affectation des utilisateurs aux boutiques.
\subsection[User stories]{User stories}
Après avoir défini le but du sprint, nous pourrons lister les user stories qui appartiennent au sprint.
La table \ref{tab:user-stories-sprint-2} représente notre backlog de Sprint 2.
\begin{table}[H]
	\centering
	\begin{tabular}{|l|l|l|}
		\hline
		\rowcolor[HTML]{C0C0C0} 
		user story &
		tâches &
		complexité \\ \hline
		\begin{tabular}[c]{@{}l@{}}En tant qu’un administrateur, je\\ souhaite d’avoir toutes les \\ informations des boutiques à jour\end{tabular} &
		Intégrer le module Boutique dans le projet &
		1 \\ \hline
		\begin{tabular}[c]{@{}l@{}}En tant qu’un administrateur, je \\ souhaite d’affecter les utilisateurs\\ CPRO aux boutiques\end{tabular} &
		\begin{tabular}[c]{@{}l@{}}ajouter les interfaces de la gestion\\ des affectations\\ consultation\\ modification\\ suppression\\ affectation\\ recherche\\ et la logique derrière\end{tabular} &
		2 \\ \hline
		\begin{tabular}[c]{@{}l@{}}En tant qu’un administrateur, je\\ souhaite d’exporter les\\  affectations des utilisateurs CPRO\\  aux boutiques\end{tabular} &
		\begin{tabular}[c]{@{}l@{}}ajouter bouton “exporter” dans\\ l’interface de gestion des\\ utilisateurs et la logique derrière\end{tabular} &
		3 \\ \hline
		\begin{tabular}[c]{@{}l@{}}En tant qu’un utilisateur de \\ Panoramix, je souhaite de simuler \\ la page SRCD\end{tabular} &
		développer une interface SRCD &
		3 \\ \hline
	\end{tabular}
	\caption{User stories de sprint 2}
	\label{tab:user-stories-sprint-2}
\end{table}









