\chapter*{Conclusion générale}
\addcontentsline{toc}{chapter}{Conclusion générale}
\markboth{Conclusion générale}{}

Ce stage, a été, sous plusieurs aspects riches d'enseignements, nous avons commencé dans un premier lieu par comprendre le contexte général de notre application et identifier les différents besoins de notre futur système. Nous avons préparé par la suite notre planning de travail en respectant les priorités de nos besoins suite à une discussion entre l'équipe. Ce projet de fin d'études a été réalisé au sein de la société Sofrecom Tunisie. Il consiste à la refonte l'application web Panoramix qui assure l'interaction entre les clients PRO PME et les conseillers réactifs.\\\newline

Le présent manuscrit détaille toutes les étapes par lesquelles nous sommes passées pour arriver au résultat attendu. Nous avons essayé tout au long de notre travail de construire notre application incrément par incrément en utilisant la méthodologie Scrum.\\\newline

Ce travail nous a été très instructif de point de vue des connaissances acquises. Il nous a procuré une opportunité pour, d'une part aborder un domaine métier et d'autre part confirmer une fois de plus nos connaissances dans le développement PHP ou les tests de non régression et toucher de près plusieurs aspects du cycle de vie d'un produit logiciel.\\\newline

Par ailleurs, d'un point de vue technologique, le projet a été très enrichissant puisqu'il nous a donné l'occasion d'étudier et d'utiliser une panoplie de technologies (Jenkins, Robot Framework, OFT, ...).\\\newline

Hormis le côté technique, ce projet a été une opportunité pour appréhender le travail dans une hiérarchie professionnelle au sein d'une grande société et, les difficultés inhérentes comme la répartition du temps et des efforts. En ainsi que les bonnes pratiques nécessaires à la réalisation d'un produit de qualité. Pour conclure, nous estimons avoir satisfait les objectifs initialement fixés, à savoir la refonte l'application web Panoramix d'Orange France.\\\newline

Ce travail a accompli ses objectifs, mais comme toute œuvre humaine il ne prétend pas la perfection et nécessite alors des améliorations. Dans cette optique, l'application peut être amélioré en changeant la solution actuelle en des micro-services. En plus, il sera toujours intéressant d'ajouter un module pour superviser les composants de système et les composants matériels en temps réel.\\