\chapter*{Annexes}
\addcontentsline{toc}{chapter}{Annexes}
\markboth{Annexes}{}
\stepcounter{chapter}
\addtocontents{lot}{\vspace{3.8mm}}
\addtocontents{lof}{\vspace{3.8mm}}

%Mettez vos annexes ici...

%===================== ANNEXE 1 =====================%
\section*{Annexe 1.~Exemple d'annexe}
\addcontentsline{toc}{section}{Annexe 1.~Exemple d'annexe}

Les chapitres doivent présenter l’essentiel du travail. Certaines informations-trop  détaillées  ou constituant un complément d’information pour toute personne qui désire mieux comprendre ou refaire une expérience décrite dans le document- peuvent être mises au niveau des annexes. Les annexes, {\bf placées après la bibliographie}, doivent donc être numérotées avec des titres (Annexe1, Annexe2, etc.).

\addcontentsline{lot}{table}{Annexe 1.1~~~Exemple tableau dans l'annexe}

Le tableau annexe 1.1 présente un exemple d'un tableau dans l'annexe.

{\raggedright \textbf{Tableau annexe 1.1:}~Exemple tableau dans l'annexe}
\begin{longtable}[c]{
    | p{.20\textwidth}
    | p{.50\textwidth} |
}
    \hline
        0 & 0 \\ \hline 
        1 & 1 \\ \hline 
        2 & 2 \\ \hline
        3 & 3 \\ \hline
        4 & 4 \\
    \hline

\end{longtable}

\newpage
%===================== ANNEXE 2 =====================%
\section*{Annexe 2.~Entreprise}
\addcontentsline{toc}{section}{Annexe 2.~Entreprise}

\addcontentsline{lof}{figure}{Annexe 2.1~~~Logo d'entreprise}

La figure annexe 2.1 présente le logo entreprise.
\begin{figure}[htpb]
    \centering
    \frame{\includegraphics[width=0.45\columnwidth]{Logo_Entreprise}}
    {\\\textbf{Figure annexe 2.1:} Logo d'entreprise}
\end{figure}

